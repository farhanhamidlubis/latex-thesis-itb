\documentclass[main.tex]{subfiles}

\begin{document}
\noindent
\chapter{Pendahuluan}
\section{Latar Belakang}
Karbonat merupakan salah satu jenis litologi yang dapat menjadi reservoir
hidrokarbon. Persentase reservoir karbonat di dunia diperkirakan mencapai 50-
60% (Burchette, 2012). Di Indonesia, reservoir karbonat memiliki porsi sekitar
50% (Park dkk., 1995) sampai dengan 80% (Merza Media dkk., 2019) dari
seluruh cekungan yang telah teridentifikasi. Angka yang cukup besar ini
memunculkan potensi penemuan hidrokarbon yang tinggi. Namun, kenyataannya,
karbonat merupakan batuan yang sangat kompleks. Karbonat memiliki struktur
internal yang sangat berbeda dengan batuan reservoir klastik pada umumnya.
Porositas pada batuan karbonat terbagi menjadi dua jenis, yaitu primer dan
sekunder. Hal ini disebabkan oleh proses pembentukan batuan karbonat yang
melibatkan proses fisika, kimia, dan biologi ketika proses seidmentasi sehingga
memiliki fasies yang heterogen disertai dengan permeabilitas yang buruk (Xu dan
Payne, 2009). Proses diagenesis juga mempersulit analisa fisika batuan dan
parameter elastis batuan serta hubungannya dengan data seismik (Zhao dkk.,
2014). Proses karakterisasi data seismik dan analisa persebaran litofasies pun
menjadi relatif lebih sulit.\\
\newline
Meskipun memiliki kompleksitas yang tinggi, sudah cukup banyak penelitian
tentang karakterisasi reservoir karbonat di Indonesia. Interpretasi kualitatif mulai
dilakukan untuk melihat persebaran reservoir karbonat menggunakan informasi
sumur dan data seismik (Alam dkk., 1999, Fainstein dan Meyer, 1997). Analisis
kuantitatif kemudian menjadi metode yang berkembang secara signifikan,
khususnya di bidang AVA dan inversi akustik, untuk memetakan distribusi
reservoir karbonat (Adriansyah dan McMechan, 2001, Pramudito dkk., 2017).
Penelitian terbaru melibatkan metode statistik neural network dalam estimasi
parameter fisik batuan (distribusi porositas dan rekahan), litofasies, serta distribusi
fluida (Hasanusi, 2014, Merza Media dkk., 2019). Meski demikian, penelitian ini
masih menggunakan inversi deterministik.\\
\newline
Salah satu faktor yang belum diperhatikan oleh penelitian-penelitian terkait
karakterisasi reservoir karbonat di Indonesia adalah unsur ketidakpastian
(uncertainty) dari hasil analisis yang diperoleh. Target dari proses inversi tidak
hanya mendapatkan model parameter yang diinginkan, namun harus bisa
menampilkan ketidakpastian dari model yang diperoleh itu sendiri (Buland dan
Omre, 2003). Ketidakpastian yang diperoleh dapat membantu penentuan
keputusan pada tahapan eksplorasi hidrokarbon selanjutnya. Dalam ranah prediksi
litofasies dan fluida (LFP) dari data seismik, ketidakpastian dapat muncul dari
banyak sumber seperti noise pengukuran dan pengolahan, aproksimasi model, dan
perubahan skala. Reservoir karbonat sendiri menambah ketidakpastian hasil
inversi dari kompleksitas fisis yang dimilikinya. Ketidakpastian tersebut, secara
statistik, dapat diatasi dengan memperbanyak data, informasi, dan pemahaman
terkait reservoir yang akan dikarakterisasi dari data seismik. Oleh karena itu,
kerangka kerja Bayesian sangat tepat dalam menggabungkan berbagai data
tersebut untuk menghitung ketidakpastian suatu proses inversi (Eidsvik dkk.,
2004, Gunning dan Glinsky, 2007, Zhao dkk., 2014).\\
\newline
Penelitian ini akan memprediksi distribusi litofasies dan konten fluida dari
reservoir karbonat menggunakan inversi Bayesian. Kerangka kerja inversi
Bayesian sangat bergantung pada parameter elastik batuan yaitu Vp, Vs, dan
densitas. Peneliti akan melakukan analisa fisika batuan terlebih dahulu untuk
mengetahui parameter elastik yang sensitif dalam mengklasifikasi reservoir
karbonat dari data seismik. Metode amplitude variation with offset (AVO)
terlinearisasi dalam kerangka Bayesian akan digunakan untuk menginversi
parameter elastik batuan dari data seismik (Buland dan Omre, 2003). Parameter
tersebut kemudian digunakan untuk melakukan inversi probabilistik Bayesian
untuk memperoleh hasil akhir berupa prediksi litofasies dan konten fluida dalam
bentuk distribusi posterior (Zhao dkk., 2014).


\section{Tujuan Penelitian}
\begin{enumerate}[itemsep = 0pt, parsep = 0pt]
    \item Menentukan parameter elastik untuk klasifikasi litoflasies dan konten fluida
    reservoir karbonat lapangan migas x;
    \item Memprediksi sebaran litofasies dan konten fluida secara probabilistik pada
    lapangan migas x;
    \item Melakukan evaluasi akurasi prediksi sebaran litofasies dan konten fluida pada
    reservoir karbonat.
\end{enumerate}
\section{Rumusah Permasalahan}
\begin{enumerate}[itemsep = 0pt, parsep = 0pt]
    \item Apa sajakah kategori litofasies dan konten fluida dari reservoir karbonat pada
    lapangan x?
    \item Apakah parameter elastik yang sensitif dalam memisahkan kelas litofasies
    dan konten fluida tersebut?
    \item Apakah inversi AVO terlinearisasi dalam kerangka dapat digunakan untuk
    mendapatkan parameter elastik dari data seismik pada lapangan x?
    \item Apakah distribusi posterior prediksi litofasies dan konten fluida dari data
    seismik memiliki kesesuaian dengan informasi pada sumur?
\end{enumerate}        

\end{document}