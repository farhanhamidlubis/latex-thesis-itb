\documentclass[main.tex]{subfiles}

\definecolor{Gray}{gray}{0.8}

\begin{document}
\chapter{Teknis Penelitian}

\section{Jadwal Penelitian}
Lampiran A

\section{Sumber Daya}
Pada penelitian ini, diperlukan data-data sebagai berikut:

\begin{center}
\begin{tabular}{l c}
    \hline
    \multicolumn{1}{c}{\textbf{Jenis Data}} & \textbf{Jumlah}\\
    \hline
    \textbf{Data seismik} (\emph{Pre-STM} dan \emph{angle gather}) 2D/3D & 1\\
    \textbf{Data sumur} & \\
    \quad Marker litologi & 2\\
    \quad Core & 2\\
    \quad Log Vp & 2\\
    \quad Log Vs & 2\\
    \quad Log densitas & 2\\
    \quad Log saturasi air & 2\\    
    \hline
\end{tabular}
\end{center}
\section{Alur Penelitian dan Luaran}
Alur penelitian secara keseluruhan dapat dilihat pada gambar di bawah ini:
% reserved for pic
Penjabaran dari masing-masing bagian pada alur penelitian di atas adalah sebagai berikut:
\begin{center}
    \begin{tabularx}{\linewidth}{c X}
        \hline
        \textbf{Tahap} & \multicolumn{1}{c}{\textbf{Metode}} \\
        \hline
        1 & Analisis Fisika Batuan - \textbf{for more than single page table, use longtable package}\\
        & \emph{Input}: \\
        & img \\
        & \emph{Output}: \\
        & img \\
        \hline
    \end{tabularx}
\end{center}

\section{Publikasi}
Penulis menggunakan tiga paper referensi utama (Buland dan Omre, 2003,
Eidsvik dkk., 2004, Zhao dkk., 2014) untuk mendapatkan rekomendasi jurnal
internasional dari penerbit Elsevier. Berikut merupakan potensi junal untuk
publikasi:

\begin{center}
    \begin{tabularx}{\linewidth}{c l c}
        \hline
        \textbf{No.} & \textbf{Nama Jurnal} & \textbf{Kuartil Scimagojr} \\
        \hline
        1 & Journal of Petroleum Science and Engineering & Q1 \\
        2 & Journal of Applied Geophysics & Q2 \\
        3 & Petroleum & Q3 \\
        4 & Interpretation & Q3 \\
        \hline
    \end{tabularx}
\end{center}

Mengingat rentang waktu penerimaan draft paper pada jurnal internasional yang
relatif lama, penulis juga berencana untuk mendaftarkan paper pada konferensi
internasional. Konferensi internasional yang dipilih minimal memberi penawaran
terbit di proceeding terindeks Scopus.


\end{document}