\documentclass[main.tex]{subfiles}

\tolerance = 1
\emergencystretch = \maxdimen
\hyphenpenalty = 10000
\hbadness = 10000

\begin{document}
% \pagenumbering{roman}
\singlespacing

\centering
\chapter*{ABSTRAK}
\addcontentsline{toc}{chapter}{ABSTRAK}
% \vspace{1\baselineskip}
{\fontsize{14}{16.8}\selectfont \textbf{
    INVERSI PROBABILISTIK DATA SEISMIK UNTUK PREDIKSI LITOFASIES DAN KONTEN FLUIDA RESERVOIR KARBONAT LAPANGAN X\\
    }
    \vspace{1\baselineskip}
    Oleh\\
    \textbf{
    Rizky Adityo Prastama\\
    NIM: 22319311\\
    (Program Studi Magister Teknik Geofisika)
    }
}
\vspace{2\baselineskip}
\justify
\noindent
Reservoir karbonat menempati posisi pertama sebagai cadangan hidrokarbon
terbesar di dunia. Akan tetapi, kompleksitas sistem pori yang dimiliki oleh batuan
sedimen non-klastik ini membuat analisis data seismik secara kuantitatif lebih
sulit dibandingkan dengan reservoir klastik konvensional. Sudah banyak studi
yang mempelajari karakteristik parameter elastik pada batuan karbonat. Salah satu
faktor yang belum dikaji secara luas adalah aspek ketidakpastian hasil pemodelan
parameter elastik itu sendiri. Informasi ketidakpastian akan sangat berguna dalam
tahapan eksplorasi migas di lapangan yang sulit seperti reservoir karbonat, baik
dari segi karakterisasi reservoir, identifikasi struktur, hingga penempatan sumur
bor. Penelitian ini bertujuan untuk melakukan prediksi litologi dan fluida reservoir
karbonat dengan menggunakan kerangka kerja Bayes. Prediksi litologi pada
kerangka kerja Bayes tidak memberikan hasil mutlak, melainkan probabilitas
litologi dan fluida berdasarkan informasi input yang digunakan. Setidaknya
terdapat dua informasi yang digunakan sebagai input yaitu data log sumur dan
parameter elastik (Vp, Vs, dan densitas) hasil inversi Bayes AVO terlinearisasi.
Data sumur akan digunakan untuk klasifikasi litologi dan fluida yang terdapat
pada zona yang akan diteliti beserta parameter elastik apa yang sensitif dalam
memisahkan kelas-kelas litologi tersebut. Parameter elastik tersebut diperoleh dari
proses inversi Bayes AVO terlinearisasi. Keunggulan kerangka kerja Bayes dalam
inversi AVO adalah hasil parameter elastik yang diperoleh merupakan nilai
dengan probabilitas posterior terbaik. Pada tahap akhir, parameter elastik ini
digunakan untuk prediksi kelas-kelas litologi dan fluida yang sudah diperoleh dari
analisa data log sumur. Masing-masing kelas litologi dan fluida tersebut akan
ditampilkan dalam bentuk penampang 2D probabilitas. Harapannya, dengan
mengetahui probabilitas dari setiap litologi dan fluida, tahapan eksplorasi
selanjutnya dapat memperhitungkan aspek ketidakpastian dari setiap keputusan
yang akan diambil.\\

\noindent
Kata kunci: karbonat, Bayes, fisika batuan, inversi AVO, probabilistik

\end{document}