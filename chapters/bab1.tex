\chapter{PENDAHULUAN}

Tulis paragraf pembuka disini (jika ada). Judul bab, yaitu Pendahuluan (ukuran 14 cetak tebal), ditulis dengan huruf kecil kecuali huruf pertama, dicetak sejajasr dengan Bab I tanpa titik di belakang huruf terakhir dan diletakkan secara simetris (\textit{centered}) pada halaman. Bab pendahuluan sedikitnya memuat (dapat dirinci dalam bentuk anak bab) hal-hal berikut:
\begin{enumerate}
\item Deskripsi topik penelitian dan latar belakang;
\item Masalah penelitian (\textit{statement of the problem}, tujuan, lkingkup permasalahan, asumsi-asumsi yang digunakan, serta hipotesis;
\item Cara pendekatan dan metode penelitian yang digunakan serta diagram alir penelitian;
\item Pelaksanaan penelitian secara garis besar;
\item Sistematika (\textit{outline}) tesis; Masalah yang hendak diselesaikan dalam tesis hendaknya dinyatakan dengan jelas, tegas, dan terinci mengingat sudah sangat menjurus dan runcingnya masalah tersebut dalam bidang spesialisasi kandidat magister.
\end{enumerate}

\section{Latar Belakang}
Jenis penuisan paragra pada naskah tesis adalah yang tidak mengandung indentasi, sehingga huruf pertama paragraf baru dimulai dari batas tepi kiri naskah dan penulisannya tidak menjorok ke dalam. Baris pertama paragraf baru dipisahkan oleh \textbf{satu baris kosong} (jarak satu setengah spasi, ukuran huruf 12) dan baris terakhir paragraf yang mendahuluinya.

Jangan memulai paragraf baru pada dasar halaman, kecuali apabila cukup tempat untuk sedikitnya dua baris. Baris terakhir sebuah paragraf jangan diletakkan pada halaman baru berikutnya, tinggalkan baris terakhir tersebut pada dasar halaman. Paragraf memuat satu pikiran utama/pokok yang tersusun dari beberapa kalimat, oleh sebab itu \textbf{hindarilah dalam satu paragraf hanya ada satu kalimat}.

\section{Masalah Penelitian}
Untuk penulis/pengarang lebih dari dua orang, yang ditulis adalah nama penulis pertama, diikuti dengan \textbf{dkk.,} kemudian tahun publikasinya. Sebagai contoh "Kramer dkk. (2005) menyatakan bahwa fosil gigi hominid yang telah ditemukan oleh timnya dari daerah Ciamis, merupakan fosil hominid pertama yang ditemukan di Jawa Barat". Selain itu bisa juga dituliskan terlebih dahulu kalimat yang disadur dari referensi kemudian menuliskan pustaka seperti pada kalimat ini (Nama Penulis, Tahun).