\documentclass{itb-thesis}
\graphicspath{{images/}}

% Redefine unit to mm
\makeatletter
\renewcommand*{\lay@value}[2]{\strip@pt\dimexpr0.351459\dimexpr\csname#2\endcsname\relax\relax mm}
\makeatother

% Input Data Tesis
\faculty{Fakultas Teknik Pertambangan dan Perminyakan}

\thesistitle{Tulis Judul Tesis Pada Bagian Ini (Jenis Huruf \textit{Times New Roman}, Huruf Kapital, Ukuran Huruf 14, Cetak Tebal, Ukuran Spasi 1)}

\fullname{Farhan Hamid Lubis}

\idnum{22319310}

\approvaldate{3 Desember 2020}

\degree{Magister Teknik Geofisika}

\monthsubmit{Desember}

\yearsubmit{2020}

\firstsupervisor{Prof. Dr. First Supervisor}
\firstnip{000001}

\secondsupervisor{Prof. Dr. Second Supervisor}
\secondnip{000002}

\thirdsupervisor{Prof. Dr. Third Supervisor}
\thirdnip{000003}

\begin{document}

% ===================================================== %
% COVER
% ===================================================== %
\cover

%\layout

% ===================================================== %
% HALAMAN PENGESAHAN
% ===================================================== %
\approvalpage


% ===================================================== %
% PEDOMAN PENGGUNAAN TESIS
% ===================================================== %
\thesisguide

Tesis Magister yang tidak dipublikasikan terdaftar dan tersedia di Perpustakaan Institut Teknologi Bandung, dan terbuka untuk umum dengan ketentuan bahwa hak cipta pada penulis dengan mengikuti aturan HaKI yang berlaku di Institut Teknologi Bandung. Referensi kepustakaan diperkenankan dicatat, tetapi pengutipan atau peringkasan hanya dapat dilakukan seizin penulis dan harus disertai dengan kaidah ilmiah untuk menyebutkan sumbernya.

\vspace{14pt}

Sitasi hasil penelitian Tesis ini dapat ditulis dalam bahasa Indonesia sebagai berikut:

\begin{hangparas}{1.27cm}{1}
Nama Belakang, Inisial Nama Depan. (Tahun): \textit{Judul tesis}, Tesis Program Magister, Institut Teknologi Bandung.
\end{hangparas}

\vspace{14pt}

dan dalam bahasa Inggris sebagai berikut:

\vspace{14pt}

\begin{hangparas}{1.27cm}{1}
Nama Belakang, Inisial Nama Depan. (Tahun): \textit{Judul tesis yang telah diterjemahkan dalam bahasa Inggris}, Master's Thesis, Institut Teknologi Bandung.
\end{hangparas}

\vspace{14pt}

Memperbanyak atau menerbitkan sebagian atau seluruh tesis haruslah seizin Dekan Sekolah Pascasarjana, Institut Teknologi Bandung.

\vspace{14pt}

Catatan: baris kedua yang merupakan kelanjutan dari baris pertama (satu judul buku), dimulai dengan 7 ketukan (satu Tab) atau ronggak (\textit{hanging indentation}: 1,27 cm) dari tepi halaman.


% ===================================================== %
% HALAMAN PERUNTUKAN
% ===================================================== %
\dedicationpage

\begin{center}

\ 
\vfill
\begin{itshape}
Halaman peruntukan (deication) bukan halaman yang diharuskan. Jika ada, pada halaman tersebut dituliskan untuk siapa tesis tersebut didedikasikan.

Kalimat pada halaman ini diposisikan di bagian tengah kertas.

Contoh

Dipersembahkan kepada orang tua, suami, anak, adik kakak, mertua serta keluarga besarku tercinta yang senantiasa mendukung lahir dan batin.
\end{itshape}
\vfill

\end{center}


% ===================================================== %
% KATA PENGANTAR
% ===================================================== %
\preface

Halaman kata pengantar dicetak pada halaman baru. Pada halaman ini mahasiswa S2 berkesempatan untuk menyatakan terima kasih secara tertulis kepada pembimbing dan perorangan lainnya yang telah memberi bimbingan, nasihat, saran dan kritik, serta kepada mereka yang telah membantu melakukan penelitian, kepada perorangan atau badan telah memberi bantuan pembiayaan, dan sebagainya.

\vspace{14pt}

Cara menulis kata pengantar beraneka ragam, tetapi semuanya hendaknya menggunakan kalimat yang baku. Ucapan terima kasih agar dibuat tidak berlebihan dan dibatasi hanya yang "\textit{scientifically related}".


% ===================================================== %
% DAFTAR ISI
% ===================================================== %
\tableofcontents
\addcontentsline{toc}{chapter}{\contentsname}


% ===================================================== %
% DAFTAR GAMBAR DAN ILUSTRASI
% ===================================================== %
\listoffigures
\addcontentsline{toc}{chapter}{\listfigurename}


% ===================================================== %
% DAFTAR TABEL
% ===================================================== %
\listoftables
\addcontentsline{toc}{chapter}{\listtablename}


% ===================================================== %
% DAFTAR SINGKATAN DAN LAMBANG
% ===================================================== %
\singkatan

\begin{singlespace}
\begin{longtable}{@{}p{3cm}@{}@{}p{8cm}@{}@{}p{3cm}@{}}

SINGKATAN & \centering Nama & \raggedright Pemakaian pertama kali pada halaman \\
\endfirsthead
LAMBANG & \centering Nama & \raggedright Pemakaian pertama kali pada halaman \\
\endhead

AMR & \textit{Adaptive Mesh Refinement} &  \tabularnewline
CT & \textit{Computed Tomography} &  \tabularnewline
DNS & Dekomposisi Nilai Singular &  \tabularnewline
HPLC & \textit{High Performance Liquid Chromatography} & \centering 10 \tabularnewline
MEH & Metode Elemen Hingga & \tabularnewline
MEHA & Metode Hingga Adaptif & \tabularnewline
MEHS & Metode Elemen Hingga Stuktur & \tabularnewline
NMR & \textit{Nuclear Magnetic Resonance} & \centering 10 \tabularnewline
PCR & \textit{Polymerase Chain Reaction} & \centering 10 \tabularnewline
RCBM & Rekonstruksi Citra Berbasis Model & \tabularnewline
Tm & Terameter & \tabularnewline
TO & Tomografi Optis & \tabularnewline
TOF & Tomografi Optis Fluoresens & \tabularnewline
\tabularnewline
LAMBANG & & \tabularnewline
\tabularnewline
$A$ & Konstanta pada hubungan tegangan & \tabularnewline
$A_1$ & Contoh simbol & \tabularnewline
$\rm A_o$ & Amplitudo sinyal sinar keluar & \tabularnewline
$\rm A_i$ & Amplitudo sinyal sinar masuk & \tabularnewline
$a$ & Vektor kerapatan foton pada suatu elemen & \tabularnewline
$a_i$ & Kecepatan & \tabularnewline
$a_{ij}$ & Fungsi reaksi variabel dalam koefisien & \tabularnewline
& persamaan diferensial & \tabularnewline
$b$ & Persamaan dasar perambatan gelombang & \tabularnewline
$c$ & Kecepetan sinar & \centering 5 \tabularnewline
$c_0$ & Gaya badan spesifik & \tabularnewline
$f$ & Peluang rapat hamburan & \centering 5 \tabularnewline
$\rm I_h$ & Iradians sinar hamburan & \tabularnewline
$L$ & Radians sinar yang menjalar & \tabularnewline
$N$ & Jumlah simpul & \tabularnewline
$n$ & Variabel bentuk area penjalaran sinar & \centering 5 \tabularnewline
$\rm n_1$ & Indeks bias medium sekitar objek & \tabularnewline
$\rm n_2$ & Indeks bias objek & \tabularnewline
$\hat{n}$ & Vektor bidang normal terhadap bidang $\delta\omega$ & \tabularnewline
$Q$ & Daya foton yang diinjeksikan per satuan & \centering 5 \tabularnewline
& volume & \tabularnewline
$\hat{S}^{n-1}$ & Posisi & \centering 5 \tabularnewline
$\hat{s}$ & Area penjalaran sinar & \centering 5 \tabularnewline
$t$ & Waktu & \centering 5 \tabularnewline
$\alpha$ & Sudut antara arah $\hat{s}$ dan $\hat{s}^\prime$& \centering 5 \tabularnewline
$\alpha_1$ & Variabel internal pertama & \tabularnewline
$\alpha_2$ & Variabel internal kedua & \tabularnewline
$\delta$ & Koefisien viskositas & \tabularnewline
$\rm \theta_a$ & Sudut masuk & \tabularnewline
$\rm \theta_b$ & Sudut keluar & \tabularnewline
$\lambda$ & Panjang gelombang & \tabularnewline
$\mu_a$ & Koefisien penyerapan & \centering 5 \tabularnewline
$\mu_s$ & Koefisien hamburan & \tabularnewline
$\Omega$ & Domain ruang suatu objek & \centering 5 \tabularnewline

\end{longtable}
\end{singlespace}

\vspace{14pt}

Catatan: Pada contoh daftar singkatan dan lambang di atas tidak semua diberi nomor halaman. Hal ini karena singkatan dan lambang tersebut tidak ada pada naskah template tesis. Berkaitan dengan hal tersebut pada naskah tesis semua singkatan dan lambang yang digunakan beserta nomor halamannya wajib ditulis dalam daftar ini. Halaman daftar singkatan dan lambang ditulis pada halaman baru. Baris-baris kata pada halaman daftar singkatan dan lambang berjarak satu spasi. Halaman ini memuat singkatan istilah, satuan dan lambang variabel/besaran (ditulis di kolom pertama), nama variabel dan nama istilah lengkap yang ditulis di belakang lambang dan singkatannya (ditulis di kolom kedua), dan nomor halaman tempat singkatan lambang muncul untuk pertama kali (ditulis di kolom ketiga).

Singktan dan lambang pada kolom pertama diurut menurut abjad Latin, huruf kapital kemudian disusul oleh huruf kecilnya, kemudian disusul dengan lambang yang ditulis dengan huruf Yunani yang juga diurut sesuai dengan abjad Yunani. Nama variabel/besaran atau nama istilah pada kolom kedua ditulis dengan huruf kecil kecuali huruf pertama yang ditulis dengan huruf kapital


% ===================================================== %
% BAB 1 - 5
% ===================================================== %
\chapter{PENDAHULUAN}

Tulis paragraf pembuka disini (jika ada). Judul bab, yaitu Pendahuluan (ukuran 14 cetak tebal), ditulis dengan huruf kecil kecuali huruf pertama, dicetak sejajasr dengan Bab I tanpa titik di belakang huruf terakhir dan diletakkan secara simetris (\textit{centered}) pada halaman. Bab pendahuluan sedikitnya memuat (dapat dirinci dalam bentuk anak bab) hal-hal berikut:
\begin{enumerate}
\item Deskripsi topik penelitian dan latar belakang;
\item Masalah penelitian (\textit{statement of the problem}, tujuan, lkingkup permasalahan, asumsi-asumsi yang digunakan, serta hipotesis;
\item Cara pendekatan dan metode penelitian yang digunakan serta diagram alir penelitian;
\item Pelaksanaan penelitian secara garis besar;
\item Sistematika (\textit{outline}) tesis; Masalah yang hendak diselesaikan dalam tesis hendaknya dinyatakan dengan jelas, tegas, dan terinci mengingat sudah sangat menjurus dan runcingnya masalah tersebut dalam bidang spesialisasi kandidat magister.
\end{enumerate}

\section{Latar Belakang}
Jenis penuisan paragra pada naskah tesis adalah yang tidak mengandung indentasi, sehingga huruf pertama paragraf baru dimulai dari batas tepi kiri naskah dan penulisannya tidak menjorok ke dalam. Baris pertama paragraf baru dipisahkan oleh \textbf{satu baris kosong} (jarak satu setengah spasi, ukuran huruf 12) dan baris terakhir paragraf yang mendahuluinya.

Jangan memulai paragraf baru pada dasar halaman, kecuali apabila cukup tempat untuk sedikitnya dua baris. Baris terakhir sebuah paragraf jangan diletakkan pada halaman baru berikutnya, tinggalkan baris terakhir tersebut pada dasar halaman. Paragraf memuat satu pikiran utama/pokok yang tersusun dari beberapa kalimat, oleh sebab itu \textbf{hindarilah dalam satu paragraf hanya ada satu kalimat}.

\section{Masalah Penelitian}
Untuk penulis/pengarang lebih dari dua orang, yang ditulis adalah nama penulis pertama, diikuti dengan \textbf{dkk.,} kemudian tahun publikasinya. Sebagai contoh "Kramer dkk. (2005) menyatakan bahwa fosil gigi hominid yang telah ditemukan oleh timnya dari daerah Ciamis, merupakan fosil hominid pertama yang ditemukan di Jawa Barat". Selain itu bisa juga dituliskan terlebih dahulu kalimat yang disadur dari referensi kemudian menuliskan pustaka seperti pada kalimat ini (Nama Penulis, Tahun).
\include{chapters/bab2}
\include{chapters/bab3}
\include{chapters/bab4}
\include{chapters/bab5}


\end{document}
