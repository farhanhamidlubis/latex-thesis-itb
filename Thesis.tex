\documentclass{itb-thesis}

% Redefine unit to mm
\makeatletter
\renewcommand*{\lay@value}[2]{\strip@pt\dimexpr0.351459\dimexpr\csname#2\endcsname\relax\relax mm}
\makeatother

% Input Data Tesis
\faculty{Fakultas Teknik Pertambangan dan Perminyakan}

\thesistitle{Tulis Judul Tesis Pada Bagian Ini (Jenis Huruf \textit{Times New Roman}, Huruf Kapital, Ukuran Huruf 14, Cetak Tebal, Ukuran Spasi 1)dfasdfasdfsadfsdfasdfasdf vdsf}

\fullname{Farhan Hamid Lubis}

\idnum{22319310}

\approvaldate{3 Desember 2020}

\degree{Magister Teknik Geofisika}

\monthsubmit{Desember}

\yearsubmit{2020}

\firstsupervisor{Prof. Dr. First Supervisor}
\firstnip{000001}

\secondsupervisor{Prof. Dr. Second Supervisor}
\secondnip{000002}

\thirdsupervisor{Prof. Dr. Third Supervisor}
\thirdnip{000003}

\begin{document}
\cover
%\layout
\approvalpage

\thesisguide
\begin{onehalfspace}
Tesis Magister yang tidak dipublikasikan terdaftar dan tersedia di Perpustakaan Institut Teknologi Bandung, dan terbuka untuk umum dengan ketentuan bahwa hak cipta pada penulis dengan mengikuti aturan HaKI yang berlaku di Institut Teknologi Bandung. Referensi kepustakaan diperkenankan dicatat, tetapi pengutipan atau peringkasan hanya dapat dilakukan seizin penulis dan harus disertai dengan kaidah ilmiah untuk menyebutkan sumbernya.

\vspace{14pt}

Sitasi hasil penelitian Tesis ini dapat ditulis dalam bahasa Indonesia sebagai berikut:

\begin{hangparas}{1.27cm}{1}
Nama Belakang, Inisial Nama Depan. (Tahun): \textit{Judul tesis}, Tesis Program Magister, Institut Teknologi Bandung.
\end{hangparas}

\vspace{14pt}

dan dalam bahasa Inggris sebagai berikut:

\vspace{14pt}

\begin{hangparas}{1.27cm}{1}
Nama Belakang, Inisial Nama Depan. (Tahun): \textit{Judul tesis yang telah diterjemahkan dalam bahasa Inggris}, Master's Thesis, Institut Teknologi Bandung.
\end{hangparas}

\vspace{14pt}

Memperbanyak atau menerbitkan sebagian atau seluruh tesis haruslah seizin Dekan Sekolah Pascasarjana, Institut Teknologi Bandung.

\vspace{14pt}

Catatan: baris kedua yang merupakan kelanjutan dari baris pertama (satu judul buku), dimulai dengan 7 ketukan (satu Tab) atau ronggak (\textit{hanging indentation}: 1,27 cm) dari tepi halaman.
\end{onehalfspace}

\dedicationpage
\begin{center}
\begin{onehalfspace}
\ 
\vfill
\begin{itshape}
Halaman peruntukan (deication) bukan halaman yang diharuskan. Jika ada, pada halaman tersebut dituliskan untuk siapa tesis tersebut didedikasikan.

Kalimat pada halaman ini diposisikan di bagian tengah kertas.

Contoh

Dipersembahkan kepada orang tua, suami, anak, adik kakak, mertua serta keluarga besarku tercinta yang senantiasa mendukung lahir dan batin.
\end{itshape}
\vfill
\end{onehalfspace}
\end{center}

\end{document}
